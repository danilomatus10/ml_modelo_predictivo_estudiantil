% informe_tecnico.tex
\documentclass[10pt, conference]{IEEEtran}
\usepackage{graphicx}
\usepackage{cite}
\usepackage{amsmath}
\usepackage{algorithm}
\usepackage{algorithmic}

\title{Modelo Predictivo de Deserción Estudiantil}
\author{
    \IEEEauthorblockN{Equipo 1}
    \IEEEauthorblockA{Universidad Ejemplo, Departamento de Ciencia de Datos}
}

\begin{document}
\maketitle

\begin{abstract}
Este trabajo presenta un modelo predictivo para identificar estudiantes en riesgo de deserción académica mediante técnicas de Machine Learning. 
Se comparan múltiples algoritmos y se aplican métodos de interpretación para validar las decisiones del modelo.
\end{abstract}

\section{Introducción}
% Contexto, motivación y objetivos

\section{Datos}
% Fuente de los datos: institución educativa
% Estructura: 31 columnas, ~4k registros
% Preprocesamiento: limpieza, Feature Engineering, SMOTE

\section{Metodología}
% Algoritmos: Random Forest, XGBoost
% Técnicas: Feature Engineering, SMOTE, GridSearchCV, BayesSearchCV
% Arquitectura del modelo: pipeline completo

\section{Resultados}
% Métricas: F1-Score 0.83, AUC-ROC 0.91
% Comparación de modelos:
\begin{tabular}{|l|c|c|c|c|}
\hline
Método & F1 & AUC & Accuracy & Recall \\
\hline
Random Forest & 0.81 & 0.88 & 0.85 & 0.78 \\
XGBoost & 0.83 & 0.91 & 0.87 & 0.82 \\
\hline
\end{tabular}

% Análisis de errores y limitaciones:

    
- Limitación: datos históricos de una sola institución
    
- No se incluyó NLP por falta de datos de texto
    
- Alto costo computacional en Bayesian Optimization


\section{Conclusión}
% Contribución: modelo predictivo con F1 0.83
% Recomendaciones: integrar el modelo en sistema académico, usar datos de más instituciones

\bibliographystyle{apa}
\begin{thebibliography}{9}
\bibitem{ref1} Cortez, P., \& Silva, A. M. G. (2008). Using data mining to predict secondary school student performance.
\bibitem{ref2} Pedregosa et al. (2011). Scikit-learn: Machine learning in Python. Journal of Machine Learning Research, 12, 2825–2830.
\end{thebibliography}
\end{document}