% informe_tecnico.tex
\documentclass[10pt]{article}
\usepackage{graphicx}
\usepackage{cite}
\usepackage{amsmath}
\usepackage[margin=1in]{geometry}

\title{Modelo Predictivo de Deserción Estudiantil}
\author{Equipo 1}
\date{\today}

\begin{document}
\maketitle

\begin{abstract}
Este trabajo presenta un modelo predictivo para identificar estudiantes en riesgo de deserción académica mediante técnicas de Machine Learning. Se comparan múltiples algoritmos y se aplican métodos de interpretación para validar las decisiones del modelo.
\end{abstract}

\section{Introducción}
% Contexto, motivación y objetivos

\section{Datos}
% Fuente de datos, estructura, preprocesamiento

\section{Metodología}
% Algoritmos usados (Random Forest, XGBoost)
% Feature Engineering
% Optimización de hiperparámetros (GridSearchCV, BayesSearchCV)

\section{Resultados}
% Métricas cuantitativas: F1-Score, AUC-ROC, Accuracy
% Comparación de modelos
% Interpretación con SHAP

\section{Conclusión}
% Contribuciones del modelo
% Recomendaciones para futuros trabajos

\bibliographystyle{apa}
\begin{thebibliography}{9}
\bibitem{ref1} Cortez, P., \& Silva, A. M. G. (2008). Using data mining to predict secondary school student performance.
\bibitem{ref2} Pedregosa et al. (2011). Scikit-learn: Machine learning in Python.
\end{thebibliography}
\end{document}